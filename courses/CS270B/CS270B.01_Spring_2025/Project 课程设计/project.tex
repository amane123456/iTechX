\documentclass[12pt]{article}
\usepackage[utf8]{inputenc}
\usepackage{graphicx}   % For figures
\usepackage{amsmath}    % For math
\usepackage{booktabs}   % For professional tables
\usepackage{hyperref}   % For hyperlinks
\usepackage[labelfont=bf]{caption} % Bold figure captions

\title{Project Title}
\author{Author Name \\ \texttt{email@example.com}}
\date{\today}

\begin{document}

\maketitle

\begin{abstract}
    Briefly summarize the problem, methodology, key results, and implications (4-5 sentences). 
\end{abstract}

\section{Problem Description \& Objectives}
\label{sec:problem}
State the problem clearly, its significance, and objectives. Example:
\begin{itemize}
    \item \textbf{Problem}: Existing methods for [task] suffer from [limitation]. 
    \item \textbf{Motivation}: Addressing this improves [application domain]. 
    \item \textbf{Objectives}: (1) Improve [metric], (2) Reduce [cost], (3) Extend to [scenario].
\end{itemize}

\section{Methodology}
\label{sec:method}
\subsection{Approach Overview}
Describe the high-level solution (1 paragraph). 

\subsection{Technical Details}
Detail the algorithms/methods. For example:
\begin{equation}
    L = \sum_{i=1}^N (y_i - f(x_i))^2 + \lambda \|\mathbf{w}\|^2 \quad \text{(Regularized Loss)}
\end{equation}

\subsection{Implementation}
\begin{itemize}
    \item \textbf{Dataset}: [Name], size, preprocessing steps.
    \item \textbf{Baselines}: Compared with [Method A, B].
    \item \textbf{Setup}: Framework (PyTorch/TensorFlow), hardware (GPU/TPU).
\end{itemize}

\section{Experimental Results}
\label{sec:results}
\subsection{Quantitative Analysis}
Present results in tables (see Table~\ref{tab:results}).

\begin{table}[h]
    \centering
    \caption{Performance comparison (higher is better).}
    \label{tab:results}
    \begin{tabular}{lcc}
        \toprule
        Method & Accuracy & F1-Score \\
        \midrule
        Baseline & 0.85 & 0.82 \\
        Ours & \textbf{0.91} & \textbf{0.89} \\
        \bottomrule
    \end{tabular}
\end{table}

\subsection{Qualitative Analysis}
Include visual results (see Figure~\ref{fig:comparison}).

\begin{figure}[h]
    \centering
    \includegraphics[width=0.8\linewidth]{example-image} % Replace with actual image
    \caption{Qualitative comparison between input (left) and output (right).}
    \label{fig:comparison}
\end{figure}

\section{Conclusions \& Future Work}
\label{sec:conclusion}
Summarize key contributions and limitations. Example:
\begin{itemize}
    \item \textbf{Conclusions}: Our method improves [metric] by [X\%] over baselines.
    \item \textbf{Limitations}: Requires [resource/computation].
    \item \textbf{Future Work}: Extend to [other domains], optimize [component].
\end{itemize}

\bibliographystyle{plain} % Choose style (e.g., IEEEtran)
\bibliography{references} % Replace with your .bib file

\end{document}