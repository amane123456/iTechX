\documentclass[11pt,oneside,a4paper]{article}
\usepackage[top=1 in, bottom=1 in, left=0.85 in, right=0.85 in]{geometry}
\usepackage{hyperref}
 \usepackage{float}
\usepackage{subcaption}

\usepackage{amsmath,amssymb,graphicx,url, algorithm2e}
\usepackage{thmtools,thm-restate,wrapfig,enumitem,mathabx}
\newtheorem{assumption}{Assumption}
\newtheorem{theorem}{Theorem}
\newtheorem{remark}{Remark}
\newtheorem{lemma}{Lemma}
\newtheorem{corollary}{Corollary}
\newtheorem{proof}{Proof}
\usepackage{color}

\newcommand{\red}[1]{\textcolor{red}{#1}}
\newcommand{\blue}[1]{\textcolor{blue}{#1}}
\newcommand{\ceil}[1]{\lceil #1 \rceil}
\DeclareMathOperator*{\argmax}{arg\,max}
\DeclareMathOperator*{\argmin}{arg\,min}


\title{Online Optimization and Learning (CS245)}
\author{Name: ~~~~~~~~~~~~ID: ~~~~~~~~~~~~Email: ~~~~~~~~~~~~}
\date{}

\begin{document}

\maketitle
\noindent
\rule{\linewidth}{0.4pt}
{\bf {\large Rules:}}
\begin{enumerate} 
    \item Deadline: \textcolor{red}{\textbf{2025/4/14/23:59:59}}.\\ The grade of the late submission subjects to the decaying policy $(75\%, 50\%, 25\%)$.
    \item Please do latex your homework and no handwriting is accepted.
    \item Submit your homework to TA(guohq@shanghaitech.edu.cn), including your PDF and Code, with filename ``{\sf name+id+CS245HW2.zip}''.
    \item \textcolor{red}{Plagiarism is not allowed.} You will fail this homework if any plagiarism is detected.
    \end{enumerate}
    \vspace{-0.3cm}
\rule{\linewidth}{0.4pt}

\newpage
\noindent  {\bf Problem 1: Explore-then-Exploit in Bandits}

\vspace{0.5cm}
\noindent 
Explore-then-Exploit is a simple and efficient algorithm for K-armed bandit problems.
\vspace{0.1in}
\hrule
\vspace{0.05in}
\noindent{\bf Explore-then-Exploit}
\vspace{0.05in}
\hrule
\vspace{0.05in}

\noindent {\bf Initialization:} Time horizon $T$, exploration times $N$.

\noindent {\bf Exploration:} \noindent In the first phase, the choice of arms does not depend on the observed rewards, and each arm is played for $N$ times.

\noindent {\bf 
Exploitation:} In all remaining rounds, the algorithm selects the arm with the highest empirical mean reward based on the exploration phase.

\vspace{0.02in}
\hrule
\vspace{0.5cm}
\noindent Consider the stochastic multi-armed bandit setting:
\begin{itemize}
    \item After $N$ times explorations, for any arm $a$,derive an upper bound on the expected estimation error $|\mu_a - \bar \mu_a|$, where $\mu_a$ is the true mean reward and $\bar \mu_a$ is its empirical mean estimate.
    \item Choose proper $N$, and prove an \textbf{upper bound} on the regret for the above algorithm.
    \item Specialize the analysis to the two-armed case and establish a \textbf{lower bound} on the regret of the algorithm.
\end{itemize}

\vspace{0.5cm}
{\noindent \bf Solution:}


\newpage
\noindent  {\bf Problem 2: Online Mirror Descent for Adversarial/Stochastic Bandits}
\vspace{0.5cm}

\noindent We have discussed Online Mirror Descent is possible to achieve good performance for Adversarial/Stochastic Bandits. 
\vspace{0.05in}
\hrule
\vspace{0.05in}
\noindent{\bf Online Mirror Descent for Adversarial/Stochastic Bandits}
\vspace{0.05in}
\hrule
\vspace{0.05in}

\noindent {\bf Initialization:} $x_1 = [1/K,...,1/K]$ and learning rate $\eta_t$. \\
\noindent For each round $t=1,\cdots, T:$ 
\begin{itemize}
\item {\bf Learner:}  Sample an arm $i$ from $x_t$.
\item {\bf Environment:} Observe the reward of arm $i$ : $r_t(i)$.
\item {\bf Estimator:} $\hat{r}_t(i) = r_t(i)/x_t(i)$ and 0 otherwise.
\item {\bf Update:} $x_{t+1} = \argmin_{x \in \mathcal K} \langle x, -\hat{r}_t \rangle + \frac{1}{\eta_t} B_{\Psi}(x;x_t).$
\end{itemize}
\vspace{0.02in}
\hrule
\vspace{0.1in}

\noindent If the regularizer $\Psi(x)$ is the negative entropy function, $B_{\psi}$ is the KL divergence and the algorithm is the classical EXP3 algorithm. 
\vspace{0.1cm}

\noindent Now we consider a different regularizer $\psi(x) = -\sum_{i=1}^K \sqrt{x_i}.$
\begin{itemize}
    \item For adversarial bandits, please provide the regret analysis of the algorithm with a proper adaptive learning rate $\eta_t$ and compare it with the regret of EXP3.
    \item For stochastic bandits, please try to provide a possible problem dependent regret analysis of the algorithm with a proper adaptive learning rate $\eta_t$.
\end{itemize}
\vspace{0.5cm}
{\noindent \bf Solution:}


\newpage
\noindent  {\bf Problem 3: Bandit Algorithms}
\vspace{0.5cm}

\noindent Consider the following protocol of Bandits problem.
\vspace{0.05in}
\hrule
\vspace{0.05in}
\noindent{\bf Learning in Bandits}
\vspace{0.05in}
\hrule
\vspace{0.05in}

\noindent {\bf Initialization:} $K$ arms. \\
\noindent For each round $t=1,\cdots, T:$ 
\begin{itemize}
\item \noindent {\bf Learner:}  Choose an arm $i \in [K].$
\item {\bf Environment:} Observe the loss of picked arm $\ell_{t,i}.$
\end{itemize}
\vspace{0.02in}
\hrule
\vspace{0.1in}

\noindent In this problem, we provide an environment with $K=32$ arms and $T=5000$ rounds, where each round you will receive a \textbf{loss} of your picked arm (note to be consistent with Homework $1,$ the environment returns the loss instead of reward). 
\vspace{0.2cm}

\noindent Let's apply the following  algorithms: 
\begin{itemize}
    \item Explore-then-exploit Algorithm in Problem $1$.
    \item UCB Algorithm: A classical algorithm for stochastic bandits. 
    \item Thompson Sampling Algorithm: A classical algorithm for stochastic bandits. 
    \item EXP3 Algorithm: A classical algorithm for adversarial bandits.
    \item Online Mirror Descent with a log-barrier regularizer.
    \item Online Mirror Descent with $\Psi(x) = -\sum_i \sqrt{x_i}$ in Problem $2$.
\end{itemize}

\noindent Like in Homework $1,$ you are supposed to choose the proper learning rates and plot the trajectories of algorithms. 
\vspace{0.2cm}

\noindent  
Please read the code sample and implement the algorithms with Python $3$. 

\vspace{0.2cm}
\noindent Note after you submit the code, we will also test your algorithm in other environments. 

\vspace{0.5cm}

\end{document}