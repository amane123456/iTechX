\documentclass[english,onecolumn]{IEEEtran}
\usepackage[T1]{fontenc}
\usepackage[colorlinks]{hyperref}
\usepackage{color,xcolor}
\usepackage{amsthm,amssymb,amsfonts,amsmath}
\usepackage{mathtools}
\usepackage{listings}
\usepackage[framed,numbered,autolinebreaks,useliterate]{mcode}
% for matlab code highlight
% load package with ``framed'' and ``numbered'' option.
\usepackage[framed,numbered,autolinebreaks,useliterate]{mcode}

\providecommand{\U}[1]{\protect\rule{.1in}{.1in}}
\topmargin            -18.0mm
\textheight           226.0mm
\oddsidemargin      -4.0mm
\textwidth            166.0mm
\def\baselinestretch{1.5}




% math operator macros
\newcommand{\madj}{\mathop{\mathrm{adj}}}
\newcommand{\trace}{\mathop{\mathrm{tr}}}


\DeclarePairedDelimiter{\abs}{\lvert}{\rvert}
\DeclarePairedDelimiter{\norm}{\lVert}{\rVert} 
\DeclarePairedDelimiter{\pnorm}{\lVert}{\rVert_p}



% special letters
\newcommand{\ba}{{\mathbf{a}}} % vector a
\newcommand{\bb}{{\mathbf{b}}} % vector b
\newcommand{\bv}{{\mathbf{v}}} % vector v
\newcommand{\be}{{\mathbf{e}}} % vector e
\newcommand{\bq}{{\mathbf{q}}} % vector q
\newcommand{\bx}{{\mathbf{x}}} % vector x
\newcommand{\bA}{{\mathbf{A}}} % matrix A
\newcommand{\y}{{\mathbf{y}}} % vector y
\newcommand{\z}{{\mathbf{z}}} % vector z
\newcommand{\0}{{\mathbf{0}}} % zero vector
\newcommand{\bH}{{\mathbf{H}}} % matrix H
\newcommand{\bI}{{\mathbf{I}}} % matrix I

\newcommand{\A}{{\mathbf{A}}} % matrix A
\newcommand{\B}{{\mathbf{B}}} % matrix B
\newcommand{\matC}{{\mathbf{C}}} % matrix C
\newcommand{\matD}{{\mathbf{D}}} % matrix D
\newcommand{\matE}{{\mathbf{E}}} % matrix E
\newcommand{\matH}{{\mathbf{H}}} % matrix H
\newcommand{\matK}{{\mathbf{K}}} % matrix K
\newcommand{\X}{{\mathbf{X}}} % matrix X
\newcommand{\Y}{{\mathbf{Y}}} % matrix Y
\newcommand{\Z}{{\mathbf{Z}}} % matrix Z
\newcommand{\M}{{\mathbf{M}}} % matrix M
\newcommand{\I}{{\mathbf{I}}} % identity matrix

\newcommand{\calV}{{\mathcal{V}}} % subspace V
\newcommand{\calU}{{\mathcal{U}}} % subspace U
\newcommand{\calS}{{\mathcal{S}}} % subspace S
\newcommand{\calM}{{\mathcal{M}}} % subspace M
\newcommand{\calN}{{\mathcal{N}}} % subspace N
\newcommand{\calR}{\mathop{\mathcal{R}}} % range
\renewcommand{\calN}{\mathop{\mathcal{N}}} % nullspace

% number fields
\newcommand{\R}{\mathbb{R}} % real numbers
\newcommand{\C}{\mathbb{C}} % complex numbers

\newcommand{\bigO}{{\mathcal{O}}} % big-Oh notation









\begin{document}



% =======================================================================================
\newpage
\noindent \textbf{Problem 3.}  (\textbf{Eigendecomposition Algorithms}, \textcolor{blue}{20 points})
\begin{enumerate}
    \item     Construct a square matrix with eigenvalues \{2,3,5,7,11\}, and then use the power iteration to find the five eigenvalues. Record the error for each iteration and plot the convergence curve (i.e. error of eigenvalue versus iteration). Matlab code is also required. (Hint: Randomly generate a matrix $\textbf{X}$ and let $\textbf{A}=\textbf{X}\Lambda\textbf{X}^{-1}$. After using the power method to find the largest eigenvalue, modify $\textbf{A}$ using $\lambda_1\textbf{v}_1\textbf{v}_1^T$, where $\textbf{v}_1$ is the eigenvector corresponding to largest eigenvalue. You can design your own stopping rules; it can be the number of iterations or convergence of eigenvalues) (\textcolor{blue}{5 points})
    
    \item Use Rayleigh Quotient iteration to find an eigenvalue of the matrix you construct in question 1). Plot the convergence curve and compare it with the curve generated by power iteration. (Note. You need to plot two curves in one figure.) Matlab code is also required. (\textcolor{blue}{5 points})

    \item Apply QR iteration, and plot the convergence curve of all eigenvalues. (Use diagonal element of matrix $\textbf{A}^{(k)}$ during iteration as eigenvalues.) Matlab code is also required. (\textcolor{blue}{10 points})
\end{enumerate}





\end{document}
