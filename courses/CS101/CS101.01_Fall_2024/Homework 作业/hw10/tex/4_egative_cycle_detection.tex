\titledquestion{Negative Cycle Detection}
The \textbf{inf($\infty$)} mentioned in this problem could be regarded as a pre-defined large enough constant. The graph mentioned in this problem is a simple directed graph. Please write down your codes in \textbf{standard C++}.

\begin{parts}
\part Consider the following implementation of the Floyd-Warshall algorithm. Suppose $W$ is the adjacency matrix of the graph, and assume that $W_{ij}=\infty$ where there is no edge between vertex $i$ and vertex $j$, and assume $W_{ii}=0$ for every vertex $i$. And other $W_{ij}$ are the weights of the edge between vertex $i$ and vertex $j$.  The array `graph' passed into the function is the adjacency matrix $W$.

\begin{cpp}
bool DetectNegCycle_Floyd(const int graph[][V])
{
    int dist[V][V];

    for (int i = 0; i < V; ++i)
        for (int j = 0; j < V; ++j)
            dist[i][j] = graph[i][j];

    for (int k = 0; k < V; ++k)
        for (int i = 0; i < V; ++i)
            for (int j = 0; j < V; ++j)
                if (dist[i][j] > dist[i][k] + dist[k][j])
                        dist[i][j] = dist[i][k] + dist[k][j];

    _____________________________________
    _____________________________________
    _____________________________________
    _____________________________________

    return false;
}
\end{cpp}

\begin{subparts}

\subpart [2] Consider the three loop lines in the Floyd-Warshall algorithm: lines $9, 10$, and $11$. Which pair(s) of these lines can be swapped without affecting the correctness of the algorithm? List all possible pairs of lines that can be swapped.

\begin{solution} \\
    \vspace{2in}

\end{solution}

\subpart[4] Add some codes in the blank lines to detect whether there are negative cycles in the graph. (You may not use all blank lines, or you can add more lines.)

\begin{solution}
\begin{cpp}
    _____________________________________
    _____________________________________
    _____________________________________
    _____________________________________
\end{cpp}
\end{solution}
\end{subparts}

\part[4] Consider the following implementation of the Bellman-Ford algorithm. The `edge' structure is used to represent the edges of the graph, with its elements $u$, $v$, and $w$ denoting an edge from node $u$ to node $v$ and its corresponding weight $w$.

\begin{cpp}
struct edge
{
    int u, v, w;
};

bool detectNegCycle_BellmanFord(const std::vector<edge>& Edge, int s)
{
    int dist[V];

    for (int i = 0; i < V; ++i)
        dist[i] = inf;
    dist[s] = 0;

    for(i = 1; i <= V - 1; ++i)
    {
        for(const auto& e : Edge)
            if (dist[e.v] > dist[e.u] + e.w)
                dist[e.v] > dist[e.u] + e.w;
    }

    _____________________________________
    _____________________________________
    _____________________________________
    _____________________________________

    return false;
}
\end{cpp}

Add some codes in the blank lines to detect whether there are negative cycles in the graph. (You may not use all blank lines, or you can add more lines.)

\begin{solution}
\begin{cpp}
    _____________________________________
    _____________________________________
    _____________________________________
    _____________________________________
\end{cpp}
\end{solution}

\end{parts}