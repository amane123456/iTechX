\titledquestion{Multiple Choices}

Each question has \textbf{one or more} correct answer(s). Select all the correct answer(s). For each question, you will get 0 points if you select one or more wrong answers, but you will get 1 point if you select a non-empty subset of the correct answers.

Write your answers in the following table.

%%%%%%%%%%%%%%%%%%%%%%%%%%%%%%%%%%%%%%%%%%%%%%%%%%%%%%%%%%%%%%%%%%%%%%%%%%%
% Note: The `LaTeX' way to answer a multiple-choice question is to replace `\choice'
% with `\choice', as what you did in the first question. However, there are still
% many students who would like to handwrite their homework. To make TA's work easier,
% you have to fill in your selected choices in the table below, no matter whether you use 
% LaTeX or not.
%%%%%%%%%%%%%%%%%%%%%%%%%%%%%%%%%%%%%%%%%%%%%%%%%%%%%%%%%%%%%%%%%%%%%%%%%%%

\begin{table}[htbp]
	\centering
	\begin{tabular}{|p{2cm}|p{2cm}|p{2cm}|p{2cm}|p{2cm}|}
		\hline 
		(a) & (b) & (c) & (d) & (e) \\
		\hline
  		%%%%%%%%%%%%%%%%%%%%%%%%%%%%%%%%%%%%%%%%%%%%%%%%%%%%%%%%%%
		% YOUR ANSWER HERE.
		   &  &  &  &  \\
            %%%%%%%%%%%%%%%%%%%%%%%%%%%%%%%%%%%%%%%%%%%%%%%%%%%%%%%%%%
		\hline
	\end{tabular} 
\end{table}

\begin{parts}


\part[2] Which of the following statement(s) is/are true for an AVL tree?
\begin{choices}
    \choice Inserting an item can unbalance non-consecutive (not directly connected) nodes on the path from the root to the inserted item before the restructuring.
    \choice Inserting an item can cause at most one node imbalanced before the restructuring.
    \choice Only at most one node-restructuring has to be performed after inserting an item.
    \choice Removing an item in leaf nodes can cause at most one node imbalanced before the restructuring. 
    
\end{choices}

\part[2] Consider an AVL tree whose height is h, which of the following is/are true? 

\begin{choices}

\choice This tree contains $\Omega(\alpha^h)$ nodes, where $\alpha = \dfrac{\sqrt{5}-1}{2}$.

\choice This tree contains $\Theta(2^h)$ nodes.

\choice This tree contains $O(h)$ nodes in the worst case.

\choice None of them above.

\end{choices}

\part[2]  Which of the following about the comparison between AVL-tree and BST is/are true? Suppose $n$ is the number of nodes in the tree.

\begin{choices}
    
    \choice The cost of searching an AVL tree is $O(\log n)$ but that of a complete binary search tree is $O(n \log n)$.

    \choice The cost of searching an AVL tree is $\Theta(\log n)$ but that of a binary search tree is $O(n)$.
    
    \choice The cost of searching a binary search tree with height h is $O(h)$ but that of an AVL tree is $O(\log n)$.
    
    \choice The corrections of both Insertion and Erasion cost $\Theta(\log n)$ time in worst cases in an AVL tree.

\end{choices}

\part[2] Which of the following statements is/are true for an AVL tree? Here one balance correction means a single rotation (in left-left or right-right cases) or a double rotation (in left-right or right-left cases). 

\begin{choices}
    \choice In an AVL tree, during the insert operation there are at most two rotations needed.
    \choice Inserting an item causes at most one node imbalance before checking if any balance correction is needed.
    \choice At most one balance correction has to be performed after inserting an item.
    \choice At most one balance correction has to be performed after removing an item.
\end{choices}

\part[2] You are given an AVL tree as a blow. Suppose we promote the minimum element in the right sub-tree when erasing a node with 2 children. Which of the following operation sequences will cause 2 imbalances that must be corrected in total in order to rebalance the tree?

\begin{tikzpicture}
\Tree
[.6
    [.3
        \edge[];[.2
        ]
        \edge[blank]; \node[blank]{};
    ]
    [.14
        \edge[];[.10
            \edge[];[.7
            ]
            \edge[];[.11
            ]
        ]
        \edge[];[.16
            \edge[blank]; \node[blank]{};
            \edge[];[.18
            ]
        ]
    ]
]
\end{tikzpicture}

\begin{choices}
    \choice Erase $2$, $6$.
    \choice Insert $4$, $5$, $12$. Erase $2$.
    \choice Erase $6$, $2$, $3$. Insert $20$.
    \choice Insert $1$, $0$, $4$, $13$, $19$.
\end{choices}

\end{parts} 

\newpage
