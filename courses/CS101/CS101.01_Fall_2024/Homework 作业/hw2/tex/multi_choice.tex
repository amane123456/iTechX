\titledquestion{Multiple Choices}

Each question has \textbf{one or more} correct answer(s). Select all the correct answer(s). For each question, you will get 0 points if you select one or more wrong answers, but you will get half points if you select a non-empty subset of the correct answers.

Write your answers in the following table.

%%%%%%%%%%%%%%%%%%%%%%%%%%%%%%%%%%%%%%%%%%%%%%%%%%%%%%%%%%%%%%%%%%%%%%%%%%%
% Note: The `LaTeX' way to answer a multiple-choices question is to replace `\choice'
% with `\choice', as what you did in the first question. However, there are still
% many students who would like to handwrite their homework. To make TA's work easier,
% you have to fill your selected choices in the table below, no matter whether you use 
% LaTeX or not.
%%%%%%%%%%%%%%%%%%%%%%%%%%%%%%%%%%%%%%%%%%%%%%%%%%%%%%%%%%%%%%%%%%%%%%%%%%%

\begin{table}[h]
    \centering
    \renewcommand{\arraystretch}{1.25}
    \begin{tabular}{|p{2cm}|p{2cm}|p{2cm}|p{2cm}|p{2cm}|p{2cm}|}
        \hline 
        (a) & (b) & (c) & (d) & (e) & (f) \\
        \hline
        % YOUR ANSWER HERE.
        &  &  &  &  &  \\

        \hline
    \end{tabular} 
\end{table}

\begin{parts}

    \part[2] Consider a table of capacity 11 using open addressing with hash function \( k \bmod 11\) and linear probing. After inserting 7 values into an empty hash table, the table is below. Which of the following choices give a possible order of the key insertion sequence?
    
    \begin{table}[h]
        \centering
        \begin{tabular}{|c|l|l|l|l|l|l|l|l|l|l|l|}
            \hline
            Index & 0 & 1 & 2 & 3 & 4  & 5  & 6 & 7 & 8 & 9 & 10\\
            \hline
            Keys  &  &  & 24 & 47 & 26  & 3 & 61 & 15 & 49 & & \\
            \hline
        \end{tabular}
    \end{table}
    
    \begin{choices}
        \choice 26, 61, 47, 3, 15, 24, 49
        \choice 26, 24, 3, 15, 61, 47, 49
        \choice 24, 61, 26, 47, 3, 15, 49
        \choice 26, 61, 15, 49, 24, 47, 3
    \end{choices}
    
    \part[2] Consider a table of capacity 8 using open addressing with hash function \( h(k) = k \bmod 8\) and probing function $H(k) = h(k) + c_1i + c_2i^2\mod 8$ with $c_1=c_2=0.5$. After inserting 7 values into an empty hash table, the table is below. Which of the following choices give a possible order of the key insertion sequence?
    \begin{table}[h]
        \centering
        \begin{tabular}{|c|l|l|l|l|l|l|l|l|l|l|l|}
            \hline
            Index & 0 & 1 & 2  & 3  & 4  & 5  & 6  & 7 \\
            \hline
            Keys  & 14 & 6 & 10 & 9 & 25 &   & 22 & 15 \\
            \hline
        \end{tabular}
    \end{table}
    \begin{choices}
        \choice 15, 22, 10, 6, 25, 14, 9
        \choice 10, 22, 15, 6, 14, 9, 25
        \choice 6, 25, 15, 14, 22, 9, 10
        \choice 10, 22, 15, 6, 14, 9, 25
    \end{choices}
    
    \part[2] Which of the following statements about the hash table are true?
    \begin{choices}
        \choice We have a hash table of size \(2n\) with a uniformly distributed hash function. If we store \(n\) elements into the hash table, then with a very high probability, there will be \textbf{no} hash collision.
        \choice In a hash table with a uniformly distributed hash function where collisions are resolved by chaining, an unsuccessful search (i.e. the required element does not exist in the table) takes \(\Theta(1)\) on average if the load factor of the hash table is \(O(1)\).
        \choice Lazy erasing means marking the entry/bin as erased rather than deleting it.
        \choice Rehashing is a technique used to resolve hash collisions.
    \end{choices}
    
    
    \part[2] Applying insertion sort and the most basic bubble sort without a flag respectively on the same array, for both algorithms, which of the following statements is/are true? (simply assume we are using swapping for insertion sort)
    \begin{choices}
        \choice There are two for-loops, which are nested within each other.
        \choice They need the same amount of element comparisons.
        \choice They need the same amount of swaps.
        \choice None of the above.
    \end{choices}
    
    \part[2] In the lecture we have learned that different sorting algorithms are suitable for different scenarios. Which of the following statements is/are suitable for insertion sort?
    \begin{choices}
        \choice Each element of the array is close to its final sorted position.
        \choice An array where only a few elements are not in its final sorted position.
        \choice A big sorted array with a small sorted array concatenated to it.
        \choice None of the above.
    \end{choices}
    
    \part[2] The time complexity for both insertion sort and bubble sort will be the same if: (assume bubble sort is flagged bubble sort)
    \begin{choices}
        \choice the input array is reversely sorted.
        \choice the input array is a list containing n copies of the same number.
        \choice the input array is already sorted.
        \choice None of the above.
    \end{choices}

\end{parts}